\documentclass[11pt]{article}

\usepackage[margin=1in]{geometry}
\usepackage{amsfonts, amsmath, amssymb}
\usepackage{fancyhdr, float, graphicx}
\usepackage[utf8]{inputenc} % Required for inputting international characters
\usepackage[T1]{fontenc} % Output font encoding for international characters
\usepackage{fouriernc} % Use the New Century Schoolbook font
\usepackage[nottoc, notlot, notlof]{tocbibind}
\usepackage{listings}
\usepackage{xcolor}
\usepackage{blindtext}
\usepackage{longtable}
\usepackage{hyperref}
\hypersetup{
	colorlinks=true,
	linkcolor=black,
	filecolor=magenta,
	urlcolor=blue,
	pdfpagemode=FullScreen,
}

\definecolor{codegreen}{rgb}{0,0.6,0}
\definecolor{codegray}{rgb}{0.5,0.5,0.5}
\definecolor{codepurple}{rgb}{0.58,0,0.82}
\definecolor{backcolour}{rgb}{0.95,0.95,0.92}

\lstdefinestyle{mystyle}{
	backgroundcolor=\color{backcolour},
	commentstyle=\color{codegreen},
	keywordstyle=\color{magenta},
	numberstyle=\tiny\color{codegray},
	stringstyle=\color{codepurple},
	basicstyle=\ttfamily\footnotesize,
	breakatwhitespace=false,
	breaklines=true,
	captionpos=b,
	keepspaces=true,
	numbers=left,
	numbersep=5pt,
	showspaces=false,
	showstringspaces=false,
	showtabs=false,
	tabsize=2
}

\lstset{style=mystyle}

% Header and Footer
\pagestyle{fancy}
\fancyhead{}
\fancyfoot{}
\fancyhead[L]{\textit{\Large{Blockchain Technologia - Fourth Year B. Tech}}}
\fancyhead[R]{\textit{Krishnaraj T}}
\fancyfoot[C]{\thepage}
\renewcommand{\footrulewidth}{1pt}

\begin{document}

\begin{titlepage}
	\centering

	%---------------------------NAMES-------------------------------

	\huge\textsc{
		MIT World Peace University
	}\\

	\vspace{0.75\baselineskip} % space after Uni Name

	\LARGE{
        Blockchain Technology\\
		Fourth Year B. Tech, Semester 8
	}

	\vfill % space after Sub Name

	%--------------------------TITLE-------------------------------

	\rule{\textwidth}{1.6pt}\vspace*{-\baselineskip}\vspace*{2pt}
	\rule{\textwidth}{0.6pt}
	\vspace{0.75\baselineskip} % Whitespace above the title

	\huge{\textsc{
        Genesis Block and Balance Transfers
        }} \\

	\vspace{0.5\baselineskip} % Whitespace below the title
	\rule{\textwidth}{0.6pt}\vspace*{-\baselineskip}\vspace*{2.8pt}
	\rule{\textwidth}{1.6pt}

	\vspace{1\baselineskip} % Whitespace after the title block

	%--------------------------SUBTITLE --------------------------	

	\LARGE\textsc{
		Lab Assignment 6
	} % Subtitle or further description
	\vfill

	%--------------------------AUTHOR-------------------------------

	Prepared By \vspace{0.5\baselineskip} % Whitespace before the editors

	\Large{
		Krishnaraj Thadesar \\
		Cyber Security and Forensics\\
        Batch A1, PA 15
	}

	\vspace{0.5\baselineskip} % Whitespace below the editor list
	\today

\end{titlepage}

\tableofcontents
\thispagestyle{empty}
\clearpage

\section{Objective}
This document aims to explain the role and creation of the genesis block in a blockchain, the initialization process, and the steps involved in transferring balances. It also explores how mining ensures the integrity and security of transactions.

\section{Theory}

\subsection{What is a Genesis Block?}
The genesis block is the first block in a blockchain. It serves as the foundation for the entire blockchain, setting the initial parameters and configurations. Unlike other blocks, the genesis block has no predecessor and is manually defined.

\subsection{Key Information in a Genesis Block}
A genesis block typically contains:
\begin{itemize}
    \item \textbf{Timestamp:} The time at which the blockchain was created.
    \item \textbf{Genesis Hash:} A unique identifier (hash) of the first block.
    \item \textbf{Merkle Root:} A hash representing the transactions in the block.
    \item \textbf{Initial State:} The starting balances of predefined addresses.
    \item \textbf{Consensus Rules:} The protocol settings such as difficulty level and block reward.
\end{itemize}

\subsection{Importance of the Genesis Block}
The genesis block is critical because it sets the initial conditions for network participants and ensures consistency in the blockchain ledger.

\section{FAQs}

\begin{enumerate}
    \item \textbf{What is a genesis block, and what role does it play in the creation of a blockchain?}
    
    The genesis block is the first block in a blockchain, serving as the foundation of the entire network. It plays a crucial role by:
    \begin{itemize}
        \item Defining the initial state of the blockchain.
        \item Establishing the cryptographic structure for all subsequent blocks.
        \item Setting consensus rules, such as mining difficulty and block rewards.
    \end{itemize}

    \item \textbf{Describe the process of creating a genesis block on a blockchain. What information is typically included in the genesis block?}
    
    Creating a genesis block involves:
    \begin{enumerate}
        \item Defining network parameters such as difficulty and block time.
        \item Creating a configuration file specifying initial balances and settings.
        \item Computing the block hash using cryptographic algorithms.
        \item Deploying the block in the blockchain client.
    \end{enumerate}
    
    Information included in a genesis block:
    \begin{itemize}
        \item Block number (usually 0).
        \item Timestamp.
        \item Parent block hash (set to 0).
        \item Initial transactions (such as pre-mined coins).
    \end{itemize}

    \item \textbf{How do you initialize the blockchain with the genesis block, and what is the significance of the block hash in this process?}
    
    The blockchain is initialized using the genesis block by:
    \begin{enumerate}
        \item Creating a configuration file containing the genesis block definition.
        \item Using a blockchain client (e.g., Ethereum Geth) to load the genesis file.
        \item Bootstrapping the network, allowing new nodes to validate blocks.
    \end{enumerate}

    The block hash is significant because:
    \begin{itemize}
        \item It serves as a unique identifier for the block.
        \item It ensures integrity by preventing tampering.
        \item It links the genesis block to future blocks in the chain.
    \end{itemize}

    \item \textbf{Walk through the steps involved in transferring balance from one address to another on a blockchain. How do the sender, recipient, and blockchain network interact during this process?}
    
    The process of transferring funds on a blockchain follows these steps:
    \begin{enumerate}
        \item The sender creates a transaction specifying the recipient’s address and the amount to be transferred.
        \item The transaction is signed using the sender’s private key.
        \item The transaction is broadcasted to the network.
        \item Miners or validators verify and include the transaction in a block.
        \item Once the block is confirmed, the recipient’s balance updates.
    \end{enumerate}
    
    During this process:
    \begin{itemize}
        \item \textbf{The sender} initiates and signs the transaction.
        \item \textbf{The recipient} waits for confirmation.
        \item \textbf{The blockchain network} validates and records the transaction.
    \end{itemize}

    \item \textbf{What is the purpose of mining or validating transactions on a blockchain, and how does it ensure the integrity and security of transactions such as balance transfers?}
    
    Mining and transaction validation ensure the security of a blockchain by:
    \begin{itemize}
        \item Verifying that transactions are legitimate.
        \item Preventing double-spending attacks.
        \item Maintaining decentralization by allowing anyone to participate.
        \item Adding transactions to the blockchain in a tamper-proof manner.
    \end{itemize}
    
    The integrity of transactions is ensured through:
    \begin{itemize}
        \item \textbf{Proof of Work (PoW):} Miners solve complex puzzles to validate transactions.
        \item \textbf{Proof of Stake (PoS):} Validators are chosen based on their staked assets.
        \item \textbf{Cryptographic Hashing:} Each block is linked to the previous one, preventing modification.
    \end{itemize}

\end{enumerate}

\section{Glossary}

\begin{itemize}
    \item \textbf{Blockchain:} A decentralized, distributed ledger that records transactions securely.
    \item \textbf{Genesis Block:} The first block in a blockchain that sets the initial state.
    \item \textbf{Block Hash:} A unique cryptographic identifier for a block.
    \item \textbf{Mining:} The process of validating transactions and adding them to the blockchain.
    \item \textbf{Consensus Mechanism:} A protocol that ensures all nodes agree on the blockchain state.
    \item \textbf{Double-Spending:} The risk of spending the same digital asset more than once.
    \item \textbf{Proof of Work (PoW):} A mining process requiring computational effort to validate transactions.
    \item \textbf{Proof of Stake (PoS):} A consensus method where validators are chosen based on their stake in the network.
\end{itemize}

\clearpage

\begin{thebibliography}{9}
    \bibitem{bitcoin} Nakamoto, S. (2008). Bitcoin: A Peer-to-Peer Electronic Cash System. Available at: \url{https://bitcoin.org/bitcoin.pdf}
    
    \bibitem{ethereum} Ethereum Developer Documentation. Available at: \url{https://ethereum.org/en/developers/}
    
    \bibitem{geth} Ethereum Geth Documentation. Available at: \url{https://geth.ethereum.org/docs/}
    
    \bibitem{blockchain} Antonopoulos, A. (2017). Mastering Bitcoin. O'Reilly Media.
    
    \bibitem{web3} Web3.js Documentation. Available at: \url{https://web3js.readthedocs.io/}
\end{thebibliography}

\end{document}