\documentclass[11pt]{article}

\usepackage[margin=1in]{geometry}
\usepackage{amsfonts, amsmath, amssymb}
\usepackage{fancyhdr, float, graphicx}
\usepackage[utf8]{inputenc} % Required for inputting international characters
\usepackage[T1]{fontenc} % Output font encoding for international characters
\usepackage{fouriernc} % Use the New Century Schoolbook font
\usepackage[nottoc, notlot, notlof]{tocbibind}
\usepackage{listings}
\usepackage{xcolor}
\usepackage{blindtext}
\usepackage{hyperref}
\hypersetup{
	colorlinks=true,
	linkcolor=black,
	filecolor=magenta,
	urlcolor=blue,
	pdfpagemode=FullScreen,
}

\definecolor{codegreen}{rgb}{0,0.6,0}
\definecolor{codegray}{rgb}{0.5,0.5,0.5}
\definecolor{codepurple}{rgb}{0.58,0,0.82}
\definecolor{backcolour}{rgb}{0.95,0.95,0.92}

\lstdefinestyle{mystyle}{
	backgroundcolor=\color{backcolour},
	commentstyle=\color{codegreen},
	keywordstyle=\color{magenta},
	numberstyle=\tiny\color{codegray},
	stringstyle=\color{codepurple},
	basicstyle=\ttfamily\footnotesize,
	breakatwhitespace=false,
	breaklines=true,
	captionpos=b,
	keepspaces=true,
	numbers=left,
	numbersep=5pt,
	showspaces=false,
	showstringspaces=false,
	showtabs=false,
	tabsize=2
}

\lstset{style=mystyle}

% Header and Footer
\pagestyle{fancy}
\fancyhead{}
\fancyfoot{}
\fancyhead[L]{\textit{\Large{Attack Reserach and Documentation - Fourth Year B. Tech}}}
\fancyhead[R]{\textit{Krishnaraj T}}
\fancyfoot[C]{\thepage}
\renewcommand{\footrulewidth}{1pt}

\begin{document}

\begin{titlepage}
	\centering

	%---------------------------NAMES-------------------------------

	\huge\textsc{
		MIT World Peace University
	}\\

	\vspace{0.75\baselineskip} % space after Uni Name

	\LARGE{
		Attack Research and Documentation\\
		Fourth Year B. Tech, Semester 8
	}

	\vfill % space after Sub Name

	%--------------------------TITLE-------------------------------

	\rule{\textwidth}{1.6pt}\vspace*{-\baselineskip}\vspace*{2pt}
	\rule{\textwidth}{0.6pt}
	\vspace{0.75\baselineskip} % Whitespace above the title

	\huge{\textsc{
        The Colonial Pipeline ransomware attack
        }} \\

	\vspace{0.5\baselineskip} % Whitespace below the title
	\rule{\textwidth}{0.6pt}\vspace*{-\baselineskip}\vspace*{2.8pt}
	\rule{\textwidth}{1.6pt}

	\vspace{1\baselineskip} % Whitespace after the title block

	%--------------------------SUBTITLE --------------------------	

	\LARGE\textsc{
		Lab Assignment 7
	} % Subtitle or further description
	\vfill

	%--------------------------AUTHOR-------------------------------

	Prepared By \vspace{0.5\baselineskip} % Whitespace before the editors

	\Large{
		Krishnaraj Thadesar \\
		Cyber Security and Forensics\\
        Batch A1, PA 15
	}

	\vspace{0.5\baselineskip} % Whitespace below the editor list
	\today

\end{titlepage}

\tableofcontents
\thispagestyle{empty}
\clearpage

\setcounter{page}{1}

\section*{1. Overview}
The Colonial Pipeline ransomware attack occurred in May 2021, targeting the largest refined oil products pipeline in the United States. The attack, attributed to the Eastern European cybercriminal group DarkSide, disrupted fuel supplies across the East Coast, affecting approximately 45\% of the region's fuel delivery system. The attackers demanded and received a ransom of 75 Bitcoin (approximately \$4.4 million USD at the time). Despite paying the ransom, Colonial Pipeline faced significant operational and reputational impacts.

\section*{2. Incident Timeline}
\begin{itemize}[leftmargin=*]
    \item \textbf{May 6, 2021:} Initial intrusion and data theft (100 GB stolen within two hours).
    \item \textbf{May 7, 2021:} Ransomware attack begins; Colonial Pipeline shuts down operations as a precaution.
    \item \textbf{May 9, 2021:} President Biden declares a state of emergency to address fuel shortages.
    \item \textbf{May 12, 2021:} Pipeline operations resume.
    \item \textbf{June 7, 2021:} Department of Justice recovers approximately 63.7 Bitcoin (\$2.3 million USD).
\end{itemize}

\section*{3. Attack Chain Analysis (MITRE ATT\&CK Mapping)}
\begin{itemize}[leftmargin=*]
    \item \textbf{Tactics:}
        \begin{itemize}
            \item Initial Access: Exploitation of Remote Services (T1210) via compromised VPN credentials.
            \item Execution: Ransomware payload deployment (Data Encrypted for Impact - T1486).
            \item Persistence: PowerShell-based backdoors and credential harvesting tools.
            \item Lateral Movement: Compromised domain administrator credentials.
            \item Exfiltration: Data exfiltrated through encrypted channels (TA0010).
        \end{itemize}
    \item \textbf{Indicators of Compromise (IOCs):}
        \begin{itemize}
            \item Malicious IPs and domains used for command-and-control.
            \item File hashes of ransomware payloads.
        \end{itemize}
\end{itemize}

\section*{4. Root Cause Analysis}
The attack originated from a compromised VPN account that lacked multi-factor authentication (MFA). The password was likely reused from a prior breach and traded on underground forums. Legacy VPN infrastructure and insufficient access controls further facilitated the breach.

\section*{5. Security Gaps and Failures}
\begin{itemize}[leftmargin=*]
    \item Lack of MFA for remote access systems.
    \item Poor network segmentation allowed lateral movement across IT systems.
    \item Inadequate monitoring and intrusion detection systems.
    \item Insufficient incident response planning and preparedness.
\end{itemize}

\section*{6. Incident Response and Mitigation}
\begin{itemize}[leftmargin=*]
    \item Paid ransom to obtain decryption key but relied on internal recovery measures for faster restoration.
    \item Engaged third-party security firm Mandiant for investigation and remediation.
    \item Shut down pipeline operations to prevent further spread of ransomware.
    \item Collaborated with federal agencies, including the FBI, to recover part of the ransom payment.
\end{itemize}

\section*{7. Impact and Consequences}
The attack caused widespread fuel shortages across the East Coast, leading to panic buying and disruptions in air travel logistics. Fuel prices rose to their highest levels since 2014. The incident highlighted vulnerabilities in critical infrastructure cybersecurity and prompted regulatory scrutiny.

\section*{8. Lessons Learned and Recommendations}
\begin{enumerate}[leftmargin=*]
    \item Implement multi-factor authentication (MFA) for all remote access systems.
    \item Regularly audit access controls and enforce least privilege principles.
    \item Enhance network segmentation to isolate critical systems from IT networks.
    \item Develop and test an incident response plan through tabletop exercises.
    \item Deploy advanced endpoint detection and response (EDR) solutions to monitor for malicious activity.
    \item Educate employees on phishing awareness and password hygiene practices.
\end{enumerate}

\clearpage
\begin{thebibliography}{99}
	\bibitem{wireshark}
	Wireshark. \\
	Website: \url{https://www.wireshark.org/}

	\bibitem{tshark}
	Tshark. \\
	Website: \url{https://www.wireshark.org/docs/man-pages/tshark.html}

	\bibitem{tcpdump}
	Tcpdump. \\
	Website: \url{https://www.tcpdump.org/}

	\bibitem{aircrack}
	AirCrack-ng. \\
	Website: \url{https://www.aircrack-ng.org/}

	\bibitem{airsnort}
	AirSnort. \\
	Website: \url{https://sourceforge.net/projects/airsnort/}
\end{thebibliography}

\end{document}