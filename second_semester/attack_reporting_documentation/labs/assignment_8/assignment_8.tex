\documentclass[11pt]{article}

\usepackage[margin=1in]{geometry}
\usepackage{amsfonts, amsmath, amssymb}
\usepackage{fancyhdr, float, graphicx}
\usepackage[utf8]{inputenc} % Required for inputting international characters
\usepackage[T1]{fontenc} % Output font encoding for international characters
\usepackage{fouriernc} % Use the New Century Schoolbook font
\usepackage[nottoc, notlot, notlof]{tocbibind}
\usepackage{listings}
\usepackage{xcolor}
\usepackage{blindtext}
\usepackage{hyperref}
\hypersetup{
	colorlinks=true,
	linkcolor=black,
	filecolor=magenta,
	urlcolor=blue,
	pdfpagemode=FullScreen,
}

\definecolor{codegreen}{rgb}{0,0.6,0}
\definecolor{codegray}{rgb}{0.5,0.5,0.5}
\definecolor{codepurple}{rgb}{0.58,0,0.82}
\definecolor{backcolour}{rgb}{0.95,0.95,0.92}

\lstdefinestyle{mystyle}{
	backgroundcolor=\color{backcolour},
	commentstyle=\color{codegreen},
	keywordstyle=\color{magenta},
	numberstyle=\tiny\color{codegray},
	stringstyle=\color{codepurple},
	basicstyle=\ttfamily\footnotesize,
	breakatwhitespace=false,
	breaklines=true,
	captionpos=b,
	keepspaces=true,
	numbers=left,
	numbersep=5pt,
	showspaces=false,
	showstringspaces=false,
	showtabs=false,
	tabsize=2
}

\lstset{style=mystyle}

% Header and Footer
\pagestyle{fancy}
\fancyhead{}
\fancyfoot{}
\fancyhead[L]{\textit{\Large{Attack Reserach and Documentation - Fourth Year B. Tech}}}
\fancyhead[R]{\textit{Krishnaraj T}}
\fancyfoot[C]{\thepage}
\renewcommand{\footrulewidth}{1pt}

\begin{document}

\begin{titlepage}
	\centering

	%---------------------------NAMES-------------------------------

	\huge\textsc{
		MIT World Peace University
	}\\

	\vspace{0.75\baselineskip} % space after Uni Name

	\LARGE{
		Attack Research and Documentation\\
		Fourth Year B. Tech, Semester 8
	}

	\vfill % space after Sub Name

	%--------------------------TITLE-------------------------------

	\rule{\textwidth}{1.6pt}\vspace*{-\baselineskip}\vspace*{2pt}
	\rule{\textwidth}{0.6pt}
	\vspace{0.75\baselineskip} % Whitespace above the title

	\huge{\textsc{
		Analyze Real-World Incident Case Studies And Document The Findings.
    }} \\

	\vspace{0.5\baselineskip} % Whitespace below the title
	\rule{\textwidth}{0.6pt}\vspace*{-\baselineskip}\vspace*{2.8pt}
	\rule{\textwidth}{1.6pt}

	\vspace{1\baselineskip} % Whitespace after the title block

	%--------------------------SUBTITLE --------------------------	

	\LARGE\textsc{
		Lab Assignment 8
	} % Subtitle or further description
	\vfill

	%--------------------------AUTHOR-------------------------------

	Prepared By \vspace{0.5\baselineskip} % Whitespace before the editors

	\Large{
		Krishnaraj Thadesar \\
		Cyber Security and Forensics\\
        Batch A1, PA 15
	}

	\vspace{0.5\baselineskip} % Whitespace below the editor list
	\today

\end{titlepage}

\tableofcontents
\thispagestyle{empty}
\clearpage

\setcounter{page}{1}

\section{Overview}
The 2021 Colonial Pipeline ransomware incident occurred on May 7, 2021, targeting the largest fuel pipeline in the United States. The attack was carried out by the DarkSide ransomware group, exploiting compromised credentials of an inactive VPN account without multi-factor authentication (MFA). This breach led to a shutdown of pipeline operations, causing widespread fuel shortages and significant financial and reputational impacts. Colonial Pipeline paid a ransom of approximately 5 million Dollars to obtain a decryption tool, though authorities later recovered 2.3 million Dollars.

\section{Incident Timeline}
\begin{itemize}
    \item \textbf{April 29, 2021:} Attackers gained initial access using compromised credentials for an inactive VPN account.
    \item \textbf{May 6, 2021:} Approximately 100 GB of data was exfiltrated by attackers.
    \item \textbf{May 7, 2021:} Ransomware was deployed, encrypting systems and prompting Colonial Pipeline to shut down operations.
    \item \textbf{May 8, 2021:} Colonial Pipeline paid a ransom of 5 million Dollars in Bitcoin to obtain a decryption tool.
    \item \textbf{May 12, 2021:} Operations resumed after partial recovery using the decryption tool.
    \item \textbf{June 7, 2021:} Authorities recovered 2.3 million Dollars of the ransom payment.
\end{itemize}

\section{Attack Chain Analysis (MITRE ATT\&CK Mapping)}
The attack can be mapped using the MITRE ATT\&CK framework:
\begin{itemize}
    \item \textbf{Initial Access (T1078 - Valid Accounts):} Compromised credentials for an inactive VPN account without MFA.
    \item \textbf{Persistence (Likely T1136 - Create Account):} Attackers maintained access through existing or new accounts.
    \item \textbf{Lateral Movement (T1021 - Remote Services):} Likely used remote desktop protocol (RDP) or similar tools for network traversal.
    \item \textbf{Data Exfiltration (T1041 - Exfiltration Over C2 Channel):} Exfiltrated approximately 100 GB of sensitive data before deploying ransomware.
    \item \textbf{Impact (T1486 - Data Encrypted for Impact):} Ransomware encrypted critical systems, disrupting operations.
\end{itemize}

Indicators of Compromise (IOCs) include encrypted file extensions and ransom notes typical of DarkSide ransomware campaigns.

\section{Root Cause Analysis}
The root cause was traced to compromised credentials for an inactive VPN account. The absence of MFA allowed attackers to gain unauthorized access. Contributing factors included poor account management practices and insufficient monitoring capabilities that failed to detect the intrusion during the week-long reconnaissance phase.

\section{Security Gaps and Failures}
The incident revealed several security gaps:
\begin{itemize}
    \item Lack of multi-factor authentication (MFA) on critical accounts.
    \item Poor account management practices, with inactive accounts still accessible.
    \item Insufficient monitoring and detection capabilities, allowing attackers to remain undetected for a week.
    \item Weak network segmentation between IT and OT networks, increasing the risk of lateral movement.
\end{itemize}

\section{Incident Response and Mitigation}
Colonial Pipeline's response included:
\begin{itemize}
    \item Immediate shutdown of pipeline operations to contain the threat.
    \item Engagement with cybersecurity firm Mandiant for investigation and recovery efforts.
    \item Payment of a 5 million Dollars ransom to obtain a decryption tool for restoring systems.
    \item Coordination with government agencies such as the FBI and Department of Energy for support and investigation.
    \item Partial recovery of ransom funds (2.3 million Dollars) by authorities in June 2021.
\end{itemize}

While the decryption tool provided by attackers was functional, it operated slowly, necessitating additional recovery measures.

\section{Impact and Consequences}
The incident had significant consequences:
\begin{itemize}
    \item Operational: A six-day shutdown caused fuel shortages across several states, leading to panic buying and price hikes.
    \item Financial: The company incurred a $5 million ransom payment, with partial recovery ($2.3 million), alongside costs associated with investigation and recovery efforts.
    \item Reputational: The breach attracted public scrutiny and regulatory attention, highlighting vulnerabilities in critical infrastructure cybersecurity practices.
\end{itemize}

The incident prompted President Biden to declare a state of emergency due to its impact on national security.

\section{Lessons Learned and Recommendations}
The Colonial Pipeline ransomware attack offers valuable lessons for improving cybersecurity:
\subsection*{Lessons Learned}
Organizations must prioritize multi-factor authentication (MFA), conduct regular audits to deactivate unused accounts, enhance monitoring capabilities for early threat detection, and implement robust network segmentation to limit lateral movement.

\subsection*{Recommendations}
Key recommendations include:
\begin{enumerate}
    \item Implement MFA on all critical access points to prevent unauthorized access.
    \item Conduct regular reviews of user accounts to deactivate unused or inactive accounts.
    \item Enhance monitoring with advanced endpoint detection tools to identify threats in real-time.
    \item Strengthen network segmentation between IT and OT environments to limit attack spread.
    \item Develop comprehensive incident response plans and conduct regular drills to ensure preparedness.
    \item Maintain isolated backups and test disaster recovery plans regularly for quick restoration after attacks.
\end{enumerate}

By adopting these measures, organizations can mitigate risks associated with ransomware attacks and enhance resilience against future threats.

\clearpage
\begin{thebibliography}{99}
	\bibitem{it_act_2000}
	Information Technology Act, 2000. \\
	Government of India. Website: \url{https://www.meity.gov.in/content/information-technology-act}

	\bibitem{mitre_attack}
	MITRE ATT\&CK Framework. \\
	Website: \url{https://attack.mitre.org/}

	\bibitem{colonial_pipeline}
	Colonial Pipeline Ransomware Attack, 2021. \\
	Website: \url{https://www.cisa.gov/news/2021/05/14/colonial-pipeline-ransomware-attack}

	\bibitem{darkside_ransomware}
	DarkSide Ransomware Group. \\
	Website: \url{https://www.fbi.gov/investigate/cyber/darkside-ransomware}

	\bibitem{mandiant_response}
	Mandiant Incident Response Services. \\
	Website: \url{https://www.mandiant.com/}

	\bibitem{cybersecurity_best_practices}
	Cybersecurity Best Practices for Critical Infrastructure. \\
	Website: \url{https://www.cisa.gov/critical-infrastructure}

\end{thebibliography}

\end{document}