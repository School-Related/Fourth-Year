\documentclass[11pt]{article}

\usepackage[margin=1in]{geometry}
\usepackage{amsfonts, amsmath, amssymb}
\usepackage{fancyhdr, float, graphicx}
\usepackage[utf8]{inputenc} % Required for inputting international characters
\usepackage[T1]{fontenc} % Output font encoding for international characters
\usepackage{fouriernc} % Use the New Century Schoolbook font
\usepackage[nottoc, notlot, notlof]{tocbibind}
\usepackage{listings}
\usepackage{xcolor}
\usepackage{blindtext}
\usepackage{hyperref}
\hypersetup{
	colorlinks=true,
	linkcolor=black,
	filecolor=magenta,
	urlcolor=blue,
	pdfpagemode=FullScreen,
}

\definecolor{codegreen}{rgb}{0,0.6,0}
\definecolor{codegray}{rgb}{0.5,0.5,0.5}
\definecolor{codepurple}{rgb}{0.58,0,0.82}
\definecolor{backcolour}{rgb}{0.95,0.95,0.92}

\lstdefinestyle{mystyle}{
	backgroundcolor=\color{backcolour},
	commentstyle=\color{codegreen},
	keywordstyle=\color{magenta},
	numberstyle=\tiny\color{codegray},
	stringstyle=\color{codepurple},
	basicstyle=\ttfamily\footnotesize,
	breakatwhitespace=false,
	breaklines=true,
	captionpos=b,
	keepspaces=true,
	numbers=left,
	numbersep=5pt,
	showspaces=false,
	showstringspaces=false,
	showtabs=false,
	tabsize=2
}

\lstset{style=mystyle}

% Header and Footer
\pagestyle{fancy}
\fancyhead{}
\fancyfoot{}
\fancyhead[L]{\textit{\Large{Attack Reserach and Documentation - Fourth Year B. Tech}}}
\fancyhead[R]{\textit{Krishnaraj T}}
\fancyfoot[C]{\thepage}
\renewcommand{\footrulewidth}{1pt}

\begin{document}

\begin{titlepage}
	\centering

	%---------------------------NAMES-------------------------------

	\huge\textsc{
		MIT World Peace University
	}\\

	\vspace{0.75\baselineskip} % space after Uni Name

	\LARGE{
		Attack Research and Documentation\\
		Fourth Year B. Tech, Semester 8
	}

	\vfill % space after Sub Name

	%--------------------------TITLE-------------------------------

	\rule{\textwidth}{1.6pt}\vspace*{-\baselineskip}\vspace*{2pt}
	\rule{\textwidth}{0.6pt}
	\vspace{0.75\baselineskip} % Whitespace above the title

	\huge{\textsc{
        Developing a Cybersecurity Policy and Documenting its Implementation
    }} \\

	\vspace{0.5\baselineskip} % Whitespace below the title
	\rule{\textwidth}{0.6pt}\vspace*{-\baselineskip}\vspace*{2.8pt}
	\rule{\textwidth}{1.6pt}

	\vspace{1\baselineskip} % Whitespace after the title block

	%--------------------------SUBTITLE --------------------------	

	\LARGE\textsc{
		Lab Assignment 7
	} % Subtitle or further description
	\vfill

	%--------------------------AUTHOR-------------------------------

	Prepared By \vspace{0.5\baselineskip} % Whitespace before the editors

	\Large{
		Krishnaraj Thadesar \\
		Cyber Security and Forensics\\
        Batch A1, PA 15
	}

	\vspace{0.5\baselineskip} % Whitespace below the editor list
	\today

\end{titlepage}

\tableofcontents
\thispagestyle{empty}
\clearpage

\setcounter{page}{1}
\section{Purpose and Scope}

\subsection{Purpose}
The purpose of this Cyber Security Policy is to safeguard the information assets of MIT World Peace University, ensure compliance with applicable laws and regulations (e.g., IT Act 2000), and maintain the integrity, confidentiality, and availability of data critical to the university’s operations, academic integrity, and reputation.

\subsection{Scope}
This policy applies to all information systems, networks, and data owned or operated by MIT World Peace University, including:
\begin{itemize}
    \item Over 1,000 dual-boot (Windows and Ubuntu) systems, 100 Macs, and smart boards with Windows installed across the campus.
    \item The Enterprise Resource Planning (ERP) system managing student data, developed by a third-party vendor.
    \item Online payment systems and all associated infrastructure.
    \item Systems and networks used by employees, students, contractors, and third parties across the university’s extensive campus and multiple buildings.
\end{itemize}

\section{Roles and Responsibilities}

\subsection{Chief Information Security Officer (CISO)}
The CISO is responsible for:
\begin{itemize}
    \item Overseeing the implementation and enforcement of this security policy.
    \item Managing cybersecurity risks and ensuring compliance with standards such as ISO 27001 and IT Act 2000.
    \item Leading the incident response team and coordinating security awareness programs.
    \item Reporting cybersecurity status to university leadership quarterly.
\end{itemize}

\subsection{System Administrators}
System Administrators are responsible for:
\begin{itemize}
    \item Maintaining the security of all systems, including dual-boot systems, Macs, and smart boards.
    \item Applying security patches and updates promptly to mitigate vulnerabilities.
    \item Monitoring systems for incidents such as keylogger installations and reporting them immediately.
    \item Managing access controls and ensuring Kaspersky antivirus is updated on all systems.
\end{itemize}

\subsection{Employees and Users}
All employees, students, and other users are responsible for:
\begin{itemize}
    \item Adhering to this policy and reporting security incidents (e.g., theft of systems or suspicious login activity).
    \item Participating in mandatory cybersecurity training and protecting their credentials.
    \item Avoiding unsafe practices, such as logging into systems with keyloggers present, and ensuring devices are secure.
\end{itemize}

\section{Risk Assessment and Management}
The university will:
\begin{itemize}
    \item Identify threats, vulnerabilities, and risks, including:
    \begin{itemize}
        \item Physical theft of systems (e.g., laptops, desktops, or Macs).
        \item Installation of keyloggers on systems used by faculty and staff.
        \item Vulnerabilities in the ERP system or online payment platforms.
        \item Exploitation of smart boards as potential attack vectors.
    \end{itemize}
    \item Conduct annual risk assessments to evaluate security controls and identify emerging risks.
    \item Implement mitigation strategies, such as physical locks for devices and advanced endpoint protection beyond Kaspersky.
\end{itemize}

\section{Access Control and Identity Management}
The university will:
\begin{itemize}
    \item Implement role-based access controls (RBAC) to limit access to systems and data (e.g., ERP student records) based on user roles.
    \item Enforce multifactor authentication (MFA) for all critical systems, including ERP, online payment portals, and faculty logins, to prevent unauthorized access via keyloggers.
    \item Require strong passwords (minimum 12 characters, mixed case, numbers, and symbols) and review access privileges quarterly.
\end{itemize}

\section{Data Protection and Privacy}
The university will:
\begin{itemize}
    \item Encrypt sensitive data, including student records in the ERP system and payment details, both at rest and in transit using industry-standard algorithms.
    \item Define data classification standards (e.g., Public, Internal, Confidential, Restricted) to categorize university data.
    \item Ensure compliance with the IT Act 2000 and other relevant privacy regulations through regular audits of data handling practices.
\end{itemize}

\section{Incident Response and Management}
The university will:
\begin{itemize}
    \item Establish an Incident Response Plan (IRP) with the following steps:
    \begin{enumerate}
        \item \textbf{Preparation:} Form an incident response team and equip it with necessary tools.
        \item \textbf{Detection and Reporting:} Monitor systems and encourage users to report incidents (e.g., stolen devices or keylogger detections).
        \item \textbf{Analysis:} Investigate incidents to assess scope and impact.
        \item \textbf{Containment:} Prevent incident escalation (e.g., isolating affected systems).
        \item \textbf{Eradication:} Remove threats (e.g., keyloggers or malware).
        \item \textbf{Recovery:} Restore systems securely.
        \item \textbf{Post-Incident Review:} Document lessons learned to improve future responses.
    \end{enumerate}
    \item Conduct cybersecurity drills biannually to test the IRP’s effectiveness.
\end{itemize}

\section{Security Audits and Compliance}
The university will:
\begin{itemize}
    \item Conduct annual security audits by external firms and quarterly vulnerability assessments to identify weaknesses.
    \item Ensure compliance with ISO 27001, IT Act 2000, and other standards through documented processes and controls.
    \item Deploy Security Information and Event Management (SIEM) tools for continuous monitoring of systems and networks.
\end{itemize}

\section{Training and Awareness}
The university will:
\begin{itemize}
    \item Conduct annual cybersecurity awareness programs for all employees and students, focusing on risks like keyloggers, phishing, and device theft.
    \item Integrate cybersecurity best practices into employee onboarding and student orientation, emphasizing secure login habits and physical device security.
    \item Run phishing simulations and distribute regular updates via email or the university intranet.
\end{itemize}

\section{Third-Party Security Management}
The university will:
\begin{itemize}
    \item Assess the security posture of third-party vendors, particularly the ERP system provider, before and during engagement.
    \item Include cybersecurity clauses in vendor contracts, mandating encryption, regular patching, and incident reporting.
    \item Audit third-party vendors annually to ensure compliance with university security standards.
\end{itemize}

\section{Policy Implementation Plan}

\subsection{Establish Governance Structure}
\begin{itemize}
    \item Appoint a CISO to lead cybersecurity efforts, reporting to the university’s executive leadership.
    \item Define roles for system administrators, security analysts, and faculty IT liaisons across multiple buildings.
    \item Form a cybersecurity steering committee with representatives from IT, administration, and academic departments.
\end{itemize}

\subsection{Develop Security Baseline Controls}
\begin{itemize}
    \item Establish minimum security standards, including mandatory Kaspersky updates, firewalls, and physical locks for systems.
    \item Configure dual-boot systems and Macs with secure boot settings and encrypted partitions.
    \item Secure smart boards with restricted network access and regular software updates.
\end{itemize}

\subsection{Monitoring and Incident Handling}
\begin{itemize}
    \item Deploy SIEM tools to monitor over 1,000 systems and smart boards for threats like keyloggers or malware.
    \item Establish an incident reporting hotline and online portal for students and staff.
    \item Train the incident response team quarterly to handle theft, malware, and ERP breaches.
\end{itemize}

\subsection{Compliance and Reporting}
\begin{itemize}
    \item Review and update the policy annually or after major incidents (e.g., widespread theft or ERP compromise).
    \item Conduct internal compliance audits biannually, focusing on ERP access logs and payment system security.
    \item Report cybersecurity metrics (e.g., incident frequency, training participation) to university leadership semiannually.
\end{itemize}

\clearpage
\begin{thebibliography}{99}
    \bibitem{it_act_2000}
    Information Technology Act, 2000. \\
    Government of India. Website: \url{https://www.meity.gov.in/content/information-technology-act}

    \bibitem{iso_27001}
    ISO/IEC 27001: Information Security Management. \\
    International Organization for Standardization. Website: \url{https://www.iso.org/isoiec-27001-information-security.html}

    \bibitem{kaspersky}
    Kaspersky Endpoint Security. \\
    Website: \url{https://www.kaspersky.com/endpoint-security}

    \bibitem{siem_tools}
    Security Information and Event Management (SIEM) Tools. \\
    Website: \url{https://www.gartner.com/en/information-technology/glossary/security-information-and-event-management-siem}

    \bibitem{erp_system}
    Enterprise Resource Planning (ERP) Systems. \\
    Website: \url{https://www.oracle.com/erp/what-is-erp.html}

    \bibitem{cybersecurity_awareness}
    Cybersecurity Awareness Training. \\
    Website: \url{https://www.cisa.gov/cybersecurity-awareness-programs}

\end{thebibliography}

\end{document}