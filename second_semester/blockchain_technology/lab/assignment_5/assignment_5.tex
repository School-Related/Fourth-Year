\documentclass[11pt]{article}

\usepackage[margin=1in]{geometry}
\usepackage{amsfonts, amsmath, amssymb}
\usepackage{fancyhdr, float, graphicx}
\usepackage[utf8]{inputenc} % Required for inputting international characters
\usepackage[T1]{fontenc} % Output font encoding for international characters
\usepackage{fouriernc} % Use the New Century Schoolbook font
\usepackage[nottoc, notlot, notlof]{tocbibind}
\usepackage{listings}
\usepackage{xcolor}
\usepackage{blindtext}
\usepackage{longtable}
\usepackage{hyperref}
\hypersetup{
	colorlinks=true,
	linkcolor=black,
	filecolor=magenta,
	urlcolor=blue,
	pdfpagemode=FullScreen,
}

\definecolor{codegreen}{rgb}{0,0.6,0}
\definecolor{codegray}{rgb}{0.5,0.5,0.5}
\definecolor{codepurple}{rgb}{0.58,0,0.82}
\definecolor{backcolour}{rgb}{0.95,0.95,0.92}

\lstdefinestyle{mystyle}{
	backgroundcolor=\color{backcolour},
	commentstyle=\color{codegreen},
	keywordstyle=\color{magenta},
	numberstyle=\tiny\color{codegray},
	stringstyle=\color{codepurple},
	basicstyle=\ttfamily\footnotesize,
	breakatwhitespace=false,
	breaklines=true,
	captionpos=b,
	keepspaces=true,
	numbers=left,
	numbersep=5pt,
	showspaces=false,
	showstringspaces=false,
	showtabs=false,
	tabsize=2
}

\lstset{style=mystyle}

% Header and Footer
\pagestyle{fancy}
\fancyhead{}
\fancyfoot{}
\fancyhead[L]{\textit{\Large{Blockchain Technologia - Fourth Year B. Tech}}}
\fancyhead[R]{\textit{Krishnaraj T}}
\fancyfoot[C]{\thepage}
\renewcommand{\footrulewidth}{1pt}

\begin{document}

\begin{titlepage}
	\centering

	%---------------------------NAMES-------------------------------

	\huge\textsc{
		MIT World Peace University
	}\\

	\vspace{0.75\baselineskip} % space after Uni Name

	\LARGE{
        Blockchain Technology\\
		Fourth Year B. Tech, Semester 8
	}

	\vfill % space after Sub Name

	%--------------------------TITLE-------------------------------

	\rule{\textwidth}{1.6pt}\vspace*{-\baselineskip}\vspace*{2pt}
	\rule{\textwidth}{0.6pt}
	\vspace{0.75\baselineskip} % Whitespace above the title

	\huge{\textsc{
        Exploring Go Ethereum (Geth)
        }} \\

	\vspace{0.5\baselineskip} % Whitespace below the title
	\rule{\textwidth}{0.6pt}\vspace*{-\baselineskip}\vspace*{2.8pt}
	\rule{\textwidth}{1.6pt}

	\vspace{1\baselineskip} % Whitespace after the title block

	%--------------------------SUBTITLE --------------------------	

	\LARGE\textsc{
		Lab Assignment 5
	} % Subtitle or further description
	\vfill

	%--------------------------AUTHOR-------------------------------

	Prepared By \vspace{0.5\baselineskip} % Whitespace before the editors

	\Large{
		Krishnaraj Thadesar \\
		Cyber Security and Forensics\\
        Batch A1, PA 15
	}

	\vspace{0.5\baselineskip} % Whitespace below the editor list
	\today

\end{titlepage}

\tableofcontents
\thispagestyle{empty}
\clearpage

\section{Objective}
This document provides a comprehensive guide to installing, configuring, and using the Geth (Go Ethereum) client to interact with the Ethereum blockchain. It covers system requirements, installation steps, account management, and network interaction.

\section{Theory}

\subsection{What is Geth?}
Geth (Go Ethereum) is an official Ethereum client implemented in the Go programming language. It allows users to run a full Ethereum node, mine Ether, deploy smart contracts, and interact with the Ethereum network.

\subsection{Why Use Geth?}
Geth is widely used for:
\begin{itemize}
    \item Running a full Ethereum node to participate in the blockchain network.
    \item Developing and testing Ethereum smart contracts.
    \item Managing Ethereum accounts and sending transactions.
    \item Deploying private Ethereum networks for development.
\end{itemize}

\subsection{Synchronization Modes in Geth}
Geth offers different synchronization modes to connect with the Ethereum blockchain:
\begin{itemize}
    \item \textbf{Full Sync:} Downloads the entire blockchain and verifies all transactions.
    \item \textbf{Fast Sync:} Downloads only recent state data while verifying historical blocks.
    \item \textbf{Light Sync:} Downloads minimal blockchain data and relies on full nodes for queries.
\end{itemize}

\section{FAQs}

\begin{enumerate}
    \item \textbf{What are the system requirements for running the Geth client?}
    
    The minimum recommended system requirements for running Geth effectively are:
    \begin{itemize}
        \item \textbf{Processor:} Dual-core CPU (Quad-core recommended)
        \item \textbf{RAM:} 4GB (8GB or more recommended)
        \item \textbf{Storage:} At least 500GB SSD (Blockchain grows over time)
        \item \textbf{Operating System:} Windows, macOS, or Linux
        \item \textbf{Internet Connection:} Stable broadband connection for syncing
    \end{itemize}

    \item \textbf{How do you install the Geth client on your system?}
    
    The installation process depends on the operating system:
    
    \textbf{On Windows:}
    \begin{enumerate}
        \item Download Geth from \href{https://geth.ethereum.org/downloads}{Geth's official website}.
        \item Run the installer and follow the setup instructions.
        \item Open Command Prompt and type \texttt{geth version} to verify the installation.
    \end{enumerate}

    \textbf{On macOS:}
    \begin{enumerate}
        \item Install Homebrew if not already installed:  
        \item Install Geth using Homebrew:  
        \texttt{brew tap ethereum/ethereum}  
        \texttt{brew install ethereum}
        \item Verify installation: \texttt{geth version}
    \end{enumerate}

    \textbf{On Linux:}
    \begin{enumerate}
        \item Download Geth from the official website or install using the package manager:
        \begin{itemize}
            \item Ubuntu/Debian: \texttt{sudo apt install geth}
            \item Arch Linux: \texttt{sudo pacman -S geth}
        \end{itemize}
        \item Verify installation: \texttt{geth version}
    \end{enumerate}

    \item \textbf{How do you initiate Geth to sync with the Ethereum blockchain?}
    
    To start syncing with the Ethereum blockchain, use the following command:

    \begin{itemize}
        \item \textbf{Fast Sync:}  
        \texttt{geth --syncmode "fast"}
        \item \textbf{Full Sync:}  
        \texttt{geth --syncmode "full"}
        \item \textbf{Light Sync:}  
        \texttt{geth --syncmode "light"}
    \end{itemize}

    Geth will start downloading blocks and updating the blockchain state.

    \item \textbf{How do you create and manage an Ethereum account using Geth?}
    
    To create a new Ethereum account:
    \begin{enumerate}
        \item Open a terminal and run:  
        \texttt{geth account new}
        \item Enter a secure passphrase when prompted.
        \item Geth generates a new account and returns an Ethereum address.
    \end{enumerate}

    To list all existing accounts:  
    \texttt{geth account list}

    To unlock an account for transactions:  
    \texttt{geth --unlock <account-address>
     --password <password-file>}

    \item \textbf{How do you interact with the Ethereum network after setting up Geth?}
    
    Once Geth is running and synced, users can:
    \begin{itemize}
        \item Send transactions:  
        \texttt{eth.sendTransaction({from: "0xYourAddress", 
        to: "0xRecipientAddress", value: web3.toWei(1, "ether")})}
        \item Check account balance:  
        \texttt{eth.getBalance("0xYourAddress")}
        \item Deploy and interact with smart contracts.
    \end{itemize}

\end{enumerate}

\section{Glossary}

\begin{itemize}
    \item \textbf{Ethereum:} A decentralized blockchain network that supports smart contracts.
    \item \textbf{Geth:} The Go Ethereum client used to run Ethereum nodes and interact with the blockchain.
    \item \textbf{Full Node:} A node that maintains the entire Ethereum blockchain history.
    \item \textbf{Light Node:} A node that downloads only block headers and relies on full nodes for queries.
    \item \textbf{Sync Mode:} The mode in which a node synchronizes with the Ethereum network (Full, Fast, or Light).
    \item \textbf{Private Key:} A secret key used to sign transactions and prove ownership of an Ethereum account.
    \item \textbf{Gas:} A measure of computational work required for Ethereum transactions and smart contract executions.
    \item \textbf{Web3.js:} A JavaScript library used to interact with the Ethereum blockchain.
\end{itemize}

\clearpage

\begin{thebibliography}{9}
    \bibitem{geth} Ethereum Geth Documentation. Available at: \url{https://geth.ethereum.org/docs/}
    
    \bibitem{ethereum} Ethereum Developer Documentation. Available at: \url{https://ethereum.org/en/developers/}
    
    \bibitem{etherscan} Etherscan - Ethereum Block Explorer. Available at: \url{https://etherscan.io/}
    
    \bibitem{web3} Web3.js Documentation. Available at: \url{https://web3js.readthedocs.io/}
    
    \bibitem{trufflesuite} Truffle Suite - Ethereum Development Framework. Available at: \url{https://trufflesuite.com/}
\end{thebibliography}


\end{document}