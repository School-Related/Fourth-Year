\documentclass[11pt]{article}

\usepackage[margin=1in]{geometry}
\usepackage{amsfonts, amsmath, amssymb}
\usepackage{fancyhdr, float, graphicx}
\usepackage[utf8]{inputenc} % Required for inputting international characters
\usepackage[T1]{fontenc} % Output font encoding for international characters
\usepackage{fouriernc} % Use the New Century Schoolbook font
\usepackage[nottoc, notlot, notlof]{tocbibind}
\usepackage{listings}
\usepackage{xcolor}
\usepackage{blindtext}
\usepackage{longtable}
\usepackage{hyperref}
\hypersetup{
	colorlinks=true,
	linkcolor=black,
	filecolor=magenta,
	urlcolor=blue,
	pdfpagemode=FullScreen,
}

\definecolor{codegreen}{rgb}{0,0.6,0}
\definecolor{codegray}{rgb}{0.5,0.5,0.5}
\definecolor{codepurple}{rgb}{0.58,0,0.82}
\definecolor{backcolour}{rgb}{0.95,0.95,0.92}

\lstdefinestyle{mystyle}{
	backgroundcolor=\color{backcolour},
	commentstyle=\color{codegreen},
	keywordstyle=\color{magenta},
	numberstyle=\tiny\color{codegray},
	stringstyle=\color{codepurple},
	basicstyle=\ttfamily\footnotesize,
	breakatwhitespace=false,
	breaklines=true,
	captionpos=b,
	keepspaces=true,
	numbers=left,
	numbersep=5pt,
	showspaces=false,
	showstringspaces=false,
	showtabs=false,
	tabsize=2
}

\lstset{style=mystyle}

% Header and Footer
\pagestyle{fancy}
\fancyhead{}
\fancyfoot{}
\fancyhead[L]{\textit{\Large{Blockchain Technologia - Fourth Year B. Tech}}}
\fancyhead[R]{\textit{Krishnaraj T}}
\fancyfoot[C]{\thepage}
\renewcommand{\footrulewidth}{1pt}

\begin{document}

\begin{titlepage}
	\centering

	%---------------------------NAMES-------------------------------

	\huge\textsc{
		MIT World Peace University
	}\\

	\vspace{0.75\baselineskip} % space after Uni Name

	\LARGE{
		Blockchain Technology\\
		Fourth Year B. Tech, Semester 8
	}

	\vfill % space after Sub Name

	%--------------------------TITLE-------------------------------

	\rule{\textwidth}{1.6pt}\vspace*{-\baselineskip}\vspace*{2pt}
	\rule{\textwidth}{0.6pt}
	\vspace{0.75\baselineskip} % Whitespace above the title

	\huge{\textsc{
			Implementing Blockchain in Python
		}} \\

	\vspace{0.5\baselineskip} % Whitespace below the title
	\rule{\textwidth}{0.6pt}\vspace*{-\baselineskip}\vspace*{2.8pt}
	\rule{\textwidth}{1.6pt}

	\vspace{1\baselineskip} % Whitespace after the title block

	%--------------------------SUBTITLE --------------------------	

	\LARGE\textsc{
		Lab Assignment 8
	} % Subtitle or further description
	\vfill

	%--------------------------AUTHOR-------------------------------

	Prepared By \vspace{0.5\baselineskip} % Whitespace before the editors

	\Large{
		Krishnaraj Thadesar \\
		Cyber Security and Forensics\\
		Batch A1, PA 15
	}

	\vspace{0.5\baselineskip} % Whitespace below the editor list
	\today

\end{titlepage}

\tableofcontents
\thispagestyle{empty}
\clearpage

\section{Objective}

The objective of this assignment is to understand the fundamentals of blockchain technology by implementing a simple blockchain using Python. This includes:  
\begin{itemize}
    \item Understanding the structure of a blockchain.
    \item Implementing blocks that store transactions.
    \item Using hashing and Proof of Work (PoW) to secure the blockchain.
    \item Validating the blockchain to ensure data integrity.
    \item Simulating basic transactions between users.
\end{itemize}


\section{Theory}

\subsection{What is a Blockchain?}
A blockchain is a decentralized digital ledger that records transactions in a series of blocks. Each block contains transaction data, a timestamp, a proof of work, and a reference to the previous block via a cryptographic hash. This structure ensures the immutability and security of data.

\subsection{Key Components of a Blockchain}
\begin{itemize}
    \item \textbf{Block:} A unit of data storage that holds transactions.
    \item \textbf{Hash:} A unique identifier generated using cryptographic algorithms to secure the block.
    \item \textbf{Proof of Work (PoW):} A computational puzzle that miners solve to validate and add new blocks to the chain.
    \item \textbf{Transactions:} Records of value transfers between users.
    \item \textbf{Chain:} A linked sequence of blocks, where each block references the previous one.
\end{itemize}

\subsection{Working of a Blockchain}
\begin{enumerate}
    \item A user initiates a transaction.
    \item The transaction is broadcasted to a network of computers (nodes).
    \item Miners validate the transaction using Proof of Work.
    \item A new block is created and added to the blockchain.
    \item The transaction is now recorded permanently.
\end{enumerate}

\subsection{Implementation in Python}
In this assignment, we implement a simple blockchain in Python with the following features:
\begin{itemize}
    \item A class-based structure for the blockchain.
    \item Functions for creating new blocks and validating the chain.
    \item A Proof of Work algorithm to secure new blocks.
    \item Transaction handling for simulating simple value transfers.
\end{itemize}

\subsection{Blockchain Security}
Blockchain ensures data integrity and security through:
\begin{itemize}
    \item \textbf{Hashing:} Prevents data tampering by linking blocks using cryptographic hashes.
    \item \textbf{Decentralization:} No single point of failure since copies exist on multiple nodes.
    \item \textbf{Consensus Mechanism:} Proof of Work ensures only valid blocks are added.
\end{itemize}



\section{Code}

\begin{lstlisting}[language=Python]
    import hashlib
    import json
    import time
    
    class Blockchain:
        def __init__(self):
            self.chain = []
            self.pending_transactions = []
            
            # Create the genesis block
            self.create_block(proof=1, previous_hash='0')
    
        def create_block(self, proof, previous_hash):
            """Creates a new block and adds it to the blockchain."""
            block = {
                'index': len(self.chain) + 1,
                'timestamp': time.time(),
                'transactions': self.pending_transactions,
                'proof': proof,
                'previous_hash': previous_hash
            }
            self.pending_transactions = []  # Reset the list of pending transactions
            self.chain.append(block)
            return block
    
        def get_previous_block(self):
            """Returns the last block in the blockchain."""
            return self.chain[-1]
    
        def proof_of_work(self, previous_proof):
            """Implements a simple Proof of Work algorithm."""
            new_proof = 1
            check_proof = False
            while not check_proof:
                hash_operation = hashlib.sha256(str(new_proof**2 - previous_proof**2).encode()).hexdigest()
                if hash_operation[:4] == '0000':  # Condition for valid proof
                    check_proof = True
                else:
                    new_proof += 1
            return new_proof
    
        def hash(self, block):
            """Creates a SHA-256 hash of a block."""
            encoded_block = json.dumps(block, sort_keys=True).encode()
            return hashlib.sha256(encoded_block).hexdigest()
    
        def add_transaction(self, sender, receiver, amount):
            """Adds a new transaction to the list of pending transactions."""
            self.pending_transactions.append({
                'sender': sender,
                'receiver': receiver,
                'amount': amount
            })
            return self.get_previous_block()['index'] + 1
    
        def is_chain_valid(self, chain):
            """Checks if the blockchain is valid."""
            previous_block = chain[0]
            index = 1
            while index < len(chain):
                block = chain[index]
                
                # Check if previous_hash matches actual hash of previous block
                if block['previous_hash'] != self.hash(previous_block):
                    return False
                
                # Check if Proof of Work is valid
                previous_proof = previous_block['proof']
                proof = block['proof']
                hash_operation = hashlib.sha256(str(proof**2 - previous_proof**2).encode()).hexdigest()
                if hash_operation[:4] != '0000':
                    return False
                
                previous_block = block
                index += 1
            return True
    
    # Test the blockchain
    if __name__ == "__main__":
        blockchain = Blockchain()
        
        # Add transactions and mine a new block
        blockchain.add_transaction(sender="Alice", receiver="Bob", amount=10)
        previous_block = blockchain.get_previous_block()
        previous_proof = previous_block['proof']
        proof = blockchain.proof_of_work(previous_proof)
        previous_hash = blockchain.hash(previous_block)
        blockchain.create_block(proof, previous_hash)
    
        # Print the blockchain
        print(json.dumps(blockchain.chain, indent=4))
    
\end{lstlisting}

\section{Output}
\begin{lstlisting}[language=bash]
[
    {
        "index": 1,
        "timestamp": 1712168400.123456,
        "transactions": [],
        "proof": 1,
        "previous_hash": "0"
    },
    {
        "index": 2,
        "timestamp": 1712168425.678901,
        "transactions": [
            {
                "sender": "Alice",
                "receiver": "Bob",
                "amount": 10
            }
        ],
        "proof": 53992,
        "previous_hash": "5d6c5e1e6e2..."
    }
]

\end{lstlisting}

\section{FAQs}

\begin{enumerate}
    \item \textbf{What is a blockchain?}  
    A blockchain is a decentralized, distributed ledger that records transactions across multiple computers in a way that ensures security, transparency, and immutability.

    \item \textbf{What is the purpose of the genesis block?}  
    The genesis block is the first block in a blockchain. It serves as the foundation for all subsequent blocks and does not have a previous hash.

    \item \textbf{How does Proof of Work (PoW) function in this blockchain?}  
    Proof of Work (PoW) requires miners to find a valid proof by solving a computational puzzle. In this implementation, a valid proof is a number that, when used in a hash operation, produces a hash with a specific number of leading zeros.

    \item \textbf{How are transactions added to a block?}  
    Transactions are first stored in a pending transactions list. When a new block is mined, these transactions are included in the block, and the list is cleared.

    \item \textbf{How is the blockchain verified?}  
    The blockchain is verified by checking that each block's previous hash matches the hash of the preceding block and that the Proof of Work conditions are met.

\end{enumerate}

\section{Glossary}

\begin{itemize}
    \item \textbf{Blockchain:} A chain of blocks containing transaction data that ensures security and transparency.
    \item \textbf{Genesis Block:} The first block in the blockchain.
    \item \textbf{Hash:} A cryptographic function that generates a unique fixed-length output for input data.
    \item \textbf{Proof of Work (PoW):} A consensus algorithm that requires solving a computational puzzle to add new blocks.
    \item \textbf{Mining:} The process of finding a valid Proof of Work to add a new block to the blockchain.
    \item \textbf{Transaction:} A record of value transfer between two parties.
    \item \textbf{Ledger:} A record-keeping system that maintains all transactions in the blockchain.
\end{itemize}

\clearpage

\begin{thebibliography}{9}

    \bibitem{blockchain} Nakamoto, S. (2008). Bitcoin: A Peer-to-Peer Electronic Cash System. Retrieved from \url{https://bitcoin.org/bitcoin.pdf}

    \bibitem{solidity} Solidity Documentation. Available at: \url{https://soliditylang.org/docs/}

    \bibitem{ethereum} Ethereum Developer Documentation. Available at: \url{https://ethereum.org/en/developers/}

    \bibitem{cryptography} Menezes, A. J., Vanstone, S. A., and Oorschot, P. C. (1996). Handbook of Applied Cryptography. CRC Press.

    \bibitem{hashing} Rivest, R. L. (1992). The MD5 Message-Digest Algorithm. RFC 1321. Retrieved from \url{https://tools.ietf.org/html/rfc1321}

\end{thebibliography}

\end{document}